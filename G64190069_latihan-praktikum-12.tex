% Options for packages loaded elsewhere
\PassOptionsToPackage{unicode}{hyperref}
\PassOptionsToPackage{hyphens}{url}
%
\documentclass[
]{article}
\title{G64190069\_latihan praktikum 12}
\author{Rizal Mujahiddan}
\date{5/17/2022}

\usepackage{amsmath,amssymb}
\usepackage{lmodern}
\usepackage{iftex}
\ifPDFTeX
  \usepackage[T1]{fontenc}
  \usepackage[utf8]{inputenc}
  \usepackage{textcomp} % provide euro and other symbols
\else % if luatex or xetex
  \usepackage{unicode-math}
  \defaultfontfeatures{Scale=MatchLowercase}
  \defaultfontfeatures[\rmfamily]{Ligatures=TeX,Scale=1}
\fi
% Use upquote if available, for straight quotes in verbatim environments
\IfFileExists{upquote.sty}{\usepackage{upquote}}{}
\IfFileExists{microtype.sty}{% use microtype if available
  \usepackage[]{microtype}
  \UseMicrotypeSet[protrusion]{basicmath} % disable protrusion for tt fonts
}{}
\makeatletter
\@ifundefined{KOMAClassName}{% if non-KOMA class
  \IfFileExists{parskip.sty}{%
    \usepackage{parskip}
  }{% else
    \setlength{\parindent}{0pt}
    \setlength{\parskip}{6pt plus 2pt minus 1pt}}
}{% if KOMA class
  \KOMAoptions{parskip=half}}
\makeatother
\usepackage{xcolor}
\IfFileExists{xurl.sty}{\usepackage{xurl}}{} % add URL line breaks if available
\IfFileExists{bookmark.sty}{\usepackage{bookmark}}{\usepackage{hyperref}}
\hypersetup{
  pdftitle={G64190069\_latihan praktikum 12},
  pdfauthor={Rizal Mujahiddan},
  hidelinks,
  pdfcreator={LaTeX via pandoc}}
\urlstyle{same} % disable monospaced font for URLs
\usepackage[margin=1in]{geometry}
\usepackage{color}
\usepackage{fancyvrb}
\newcommand{\VerbBar}{|}
\newcommand{\VERB}{\Verb[commandchars=\\\{\}]}
\DefineVerbatimEnvironment{Highlighting}{Verbatim}{commandchars=\\\{\}}
% Add ',fontsize=\small' for more characters per line
\usepackage{framed}
\definecolor{shadecolor}{RGB}{248,248,248}
\newenvironment{Shaded}{\begin{snugshade}}{\end{snugshade}}
\newcommand{\AlertTok}[1]{\textcolor[rgb]{0.94,0.16,0.16}{#1}}
\newcommand{\AnnotationTok}[1]{\textcolor[rgb]{0.56,0.35,0.01}{\textbf{\textit{#1}}}}
\newcommand{\AttributeTok}[1]{\textcolor[rgb]{0.77,0.63,0.00}{#1}}
\newcommand{\BaseNTok}[1]{\textcolor[rgb]{0.00,0.00,0.81}{#1}}
\newcommand{\BuiltInTok}[1]{#1}
\newcommand{\CharTok}[1]{\textcolor[rgb]{0.31,0.60,0.02}{#1}}
\newcommand{\CommentTok}[1]{\textcolor[rgb]{0.56,0.35,0.01}{\textit{#1}}}
\newcommand{\CommentVarTok}[1]{\textcolor[rgb]{0.56,0.35,0.01}{\textbf{\textit{#1}}}}
\newcommand{\ConstantTok}[1]{\textcolor[rgb]{0.00,0.00,0.00}{#1}}
\newcommand{\ControlFlowTok}[1]{\textcolor[rgb]{0.13,0.29,0.53}{\textbf{#1}}}
\newcommand{\DataTypeTok}[1]{\textcolor[rgb]{0.13,0.29,0.53}{#1}}
\newcommand{\DecValTok}[1]{\textcolor[rgb]{0.00,0.00,0.81}{#1}}
\newcommand{\DocumentationTok}[1]{\textcolor[rgb]{0.56,0.35,0.01}{\textbf{\textit{#1}}}}
\newcommand{\ErrorTok}[1]{\textcolor[rgb]{0.64,0.00,0.00}{\textbf{#1}}}
\newcommand{\ExtensionTok}[1]{#1}
\newcommand{\FloatTok}[1]{\textcolor[rgb]{0.00,0.00,0.81}{#1}}
\newcommand{\FunctionTok}[1]{\textcolor[rgb]{0.00,0.00,0.00}{#1}}
\newcommand{\ImportTok}[1]{#1}
\newcommand{\InformationTok}[1]{\textcolor[rgb]{0.56,0.35,0.01}{\textbf{\textit{#1}}}}
\newcommand{\KeywordTok}[1]{\textcolor[rgb]{0.13,0.29,0.53}{\textbf{#1}}}
\newcommand{\NormalTok}[1]{#1}
\newcommand{\OperatorTok}[1]{\textcolor[rgb]{0.81,0.36,0.00}{\textbf{#1}}}
\newcommand{\OtherTok}[1]{\textcolor[rgb]{0.56,0.35,0.01}{#1}}
\newcommand{\PreprocessorTok}[1]{\textcolor[rgb]{0.56,0.35,0.01}{\textit{#1}}}
\newcommand{\RegionMarkerTok}[1]{#1}
\newcommand{\SpecialCharTok}[1]{\textcolor[rgb]{0.00,0.00,0.00}{#1}}
\newcommand{\SpecialStringTok}[1]{\textcolor[rgb]{0.31,0.60,0.02}{#1}}
\newcommand{\StringTok}[1]{\textcolor[rgb]{0.31,0.60,0.02}{#1}}
\newcommand{\VariableTok}[1]{\textcolor[rgb]{0.00,0.00,0.00}{#1}}
\newcommand{\VerbatimStringTok}[1]{\textcolor[rgb]{0.31,0.60,0.02}{#1}}
\newcommand{\WarningTok}[1]{\textcolor[rgb]{0.56,0.35,0.01}{\textbf{\textit{#1}}}}
\usepackage{graphicx}
\makeatletter
\def\maxwidth{\ifdim\Gin@nat@width>\linewidth\linewidth\else\Gin@nat@width\fi}
\def\maxheight{\ifdim\Gin@nat@height>\textheight\textheight\else\Gin@nat@height\fi}
\makeatother
% Scale images if necessary, so that they will not overflow the page
% margins by default, and it is still possible to overwrite the defaults
% using explicit options in \includegraphics[width, height, ...]{}
\setkeys{Gin}{width=\maxwidth,height=\maxheight,keepaspectratio}
% Set default figure placement to htbp
\makeatletter
\def\fps@figure{htbp}
\makeatother
\setlength{\emergencystretch}{3em} % prevent overfull lines
\providecommand{\tightlist}{%
  \setlength{\itemsep}{0pt}\setlength{\parskip}{0pt}}
\setcounter{secnumdepth}{-\maxdimen} % remove section numbering
\ifLuaTeX
  \usepackage{selnolig}  % disable illegal ligatures
\fi

\begin{document}
\maketitle

\begin{Shaded}
\begin{Highlighting}[]
\FunctionTok{library}\NormalTok{(readxl)}
\FunctionTok{library}\NormalTok{(EnvStats)}
\end{Highlighting}
\end{Shaded}

\begin{verbatim}
## 
## Attaching package: 'EnvStats'
\end{verbatim}

\begin{verbatim}
## The following objects are masked from 'package:stats':
## 
##     predict, predict.lm
\end{verbatim}

\begin{verbatim}
## The following object is masked from 'package:base':
## 
##     print.default
\end{verbatim}

\begin{Shaded}
\begin{Highlighting}[]
\FunctionTok{library}\NormalTok{(MASS)}
\end{Highlighting}
\end{Shaded}

\begin{verbatim}
## 
## Attaching package: 'MASS'
\end{verbatim}

\begin{verbatim}
## The following object is masked from 'package:EnvStats':
## 
##     boxcox
\end{verbatim}

\begin{Shaded}
\begin{Highlighting}[]
\FunctionTok{library}\NormalTok{(fitdistrplus)}
\end{Highlighting}
\end{Shaded}

\begin{verbatim}
## Loading required package: survival
\end{verbatim}

\begin{Shaded}
\begin{Highlighting}[]
\FunctionTok{library}\NormalTok{(ggplot2)}
\FunctionTok{library}\NormalTok{(psych)}
\end{Highlighting}
\end{Shaded}

\begin{verbatim}
## 
## Attaching package: 'psych'
\end{verbatim}

\begin{verbatim}
## The following objects are masked from 'package:ggplot2':
## 
##     %+%, alpha
\end{verbatim}

\begin{Shaded}
\begin{Highlighting}[]
\FunctionTok{library}\NormalTok{(car)}
\end{Highlighting}
\end{Shaded}

\begin{verbatim}
## Loading required package: carData
\end{verbatim}

\begin{verbatim}
## 
## Attaching package: 'car'
\end{verbatim}

\begin{verbatim}
## The following object is masked from 'package:psych':
## 
##     logit
\end{verbatim}

\begin{verbatim}
## The following object is masked from 'package:EnvStats':
## 
##     qqPlot
\end{verbatim}

\hypertarget{nomor-1}{%
\section{Nomor 1}\label{nomor-1}}

Untuk Nomor 1, Dipergunakan methode mendeteksi pencilan dengan
eksplorasi data dengan menggunakan boxplot dengan IQR = 1.5

\begin{Shaded}
\begin{Highlighting}[]
\NormalTok{data\_sheet1 }\OtherTok{\textless{}{-}} \FunctionTok{read\_excel}\NormalTok{(}\StringTok{"Data praktikum 12.xlsx"}\NormalTok{,}\AttributeTok{sheet=}\StringTok{"data 100"}\NormalTok{)}
\NormalTok{data\_sheet2 }\OtherTok{\textless{}{-}} \FunctionTok{read\_excel}\NormalTok{(}\StringTok{"Data praktikum 12.xlsx"}\NormalTok{,}\AttributeTok{sheet=}\StringTok{"data 10"}\NormalTok{)}

\NormalTok{x1 }\OtherTok{\textless{}{-}}\NormalTok{ data\_sheet2}\SpecialCharTok{$}\NormalTok{x1}
\NormalTok{x2 }\OtherTok{\textless{}{-}}\NormalTok{ data\_sheet2}\SpecialCharTok{$}\NormalTok{x2}
\NormalTok{y1 }\OtherTok{\textless{}{-}}\NormalTok{ data\_sheet1}\SpecialCharTok{$}\NormalTok{y1}
\NormalTok{y2 }\OtherTok{\textless{}{-}}\NormalTok{ data\_sheet1}\SpecialCharTok{$}\NormalTok{y2}


\FunctionTok{par}\NormalTok{(}\AttributeTok{mfrow=}\FunctionTok{c}\NormalTok{(}\DecValTok{1}\NormalTok{,}\DecValTok{2}\NormalTok{))}
\FunctionTok{boxplot}\NormalTok{(x1,}\AttributeTok{main=}\StringTok{"x1"}\NormalTok{)}
\FunctionTok{boxplot}\NormalTok{(x2,}\AttributeTok{main=}\StringTok{"x2"}\NormalTok{)}
\end{Highlighting}
\end{Shaded}

\includegraphics{G64190069_latihan-praktikum-12_files/figure-latex/unnamed-chunk-2-1.pdf}
\#\# X1 ada satu outlier yang terlihat, kemudian jika diperhatikan
kembali, maka hasilnya cenderung menjulur ke kanan

\hypertarget{rosnertest}{%
\subsubsection{rosnerTest}\label{rosnertest}}

\begin{Shaded}
\begin{Highlighting}[]
\NormalTok{ros\_x1 }\OtherTok{\textless{}{-}} \FunctionTok{rosnerTest}\NormalTok{(x1,}\AttributeTok{k=}\DecValTok{4}\NormalTok{)}
\end{Highlighting}
\end{Shaded}

\begin{verbatim}
## Warning in rosnerTest(x1, k = 4): The true Type I error may be larger than
## assumed. See the help file for 'rosnerTest' for a table with information on the
## estimated Type I error level.
\end{verbatim}

\begin{Shaded}
\begin{Highlighting}[]
\NormalTok{allstat\_ros }\OtherTok{\textless{}{-}}\NormalTok{ ros\_x1}\SpecialCharTok{$}\NormalTok{all.stats}
\NormalTok{vec\_x1\_ros }\OtherTok{\textless{}{-}}\NormalTok{ x1[allstat\_ros[allstat\_ros[}\StringTok{\textquotesingle{}Outlier\textquotesingle{}}\NormalTok{]}\SpecialCharTok{==}\ConstantTok{TRUE}\NormalTok{,}\StringTok{\textquotesingle{}Obs.Num\textquotesingle{}}\NormalTok{]]}
\NormalTok{vec\_x1\_ros}
\end{Highlighting}
\end{Shaded}

\begin{verbatim}
## [1] 24.75342
\end{verbatim}

\hypertarget{x2}{%
\subsection{X2}\label{x2}}

Tidak outlier yang terlihat, Kemudian Disini maka cenderung menyebar
normal tetapi mean agak bergeser ke kiri

\begin{Shaded}
\begin{Highlighting}[]
\NormalTok{ros\_x2 }\OtherTok{\textless{}{-}} \FunctionTok{rosnerTest}\NormalTok{(x2,}\AttributeTok{k=}\DecValTok{5}\NormalTok{)}
\end{Highlighting}
\end{Shaded}

\begin{verbatim}
## Warning in rosnerTest(x2, k = 5): The true Type I error may be larger than
## assumed. See the help file for 'rosnerTest' for a table with information on the
## estimated Type I error level.
\end{verbatim}

\begin{Shaded}
\begin{Highlighting}[]
\NormalTok{allstat\_ros }\OtherTok{\textless{}{-}}\NormalTok{ ros\_x2}\SpecialCharTok{$}\NormalTok{all.stats}
\NormalTok{vec\_x2\_ros }\OtherTok{\textless{}{-}}\NormalTok{ x2[allstat\_ros[allstat\_ros[}\StringTok{\textquotesingle{}Outlier\textquotesingle{}}\NormalTok{]}\SpecialCharTok{==}\ConstantTok{TRUE}\NormalTok{,}\StringTok{\textquotesingle{}Obs.Num\textquotesingle{}}\NormalTok{]]}
\NormalTok{vec\_x2\_ros}
\end{Highlighting}
\end{Shaded}

\begin{verbatim}
## [1]  7.526452 -2.991378  4.533626  3.939124
\end{verbatim}

\begin{Shaded}
\begin{Highlighting}[]
\FunctionTok{par}\NormalTok{(}\AttributeTok{mfrow=}\FunctionTok{c}\NormalTok{(}\DecValTok{1}\NormalTok{,}\DecValTok{2}\NormalTok{))}
\FunctionTok{boxplot}\NormalTok{(y1,}\AttributeTok{main=}\StringTok{"y1"}\NormalTok{)}
\FunctionTok{boxplot}\NormalTok{(y2,}\AttributeTok{main=}\StringTok{"y2"}\NormalTok{)}
\end{Highlighting}
\end{Shaded}

\includegraphics{G64190069_latihan-praktikum-12_files/figure-latex/unnamed-chunk-5-1.pdf}

\hypertarget{y1}{%
\subsection{Y1}\label{y1}}

ada satu outlier yang terlihat, kemudian jika diperhatikan kembali,
median ada di sebelah kenan

\begin{Shaded}
\begin{Highlighting}[]
\NormalTok{ros\_y1 }\OtherTok{\textless{}{-}} \FunctionTok{rosnerTest}\NormalTok{(y1,}\AttributeTok{k=}\DecValTok{5}\NormalTok{)}
\NormalTok{allstat\_ros }\OtherTok{\textless{}{-}}\NormalTok{ ros\_y1}\SpecialCharTok{$}\NormalTok{all.stats}
\NormalTok{vec\_y1\_ros }\OtherTok{\textless{}{-}}\NormalTok{ y1[allstat\_ros[allstat\_ros[}\StringTok{\textquotesingle{}Outlier\textquotesingle{}}\NormalTok{]}\SpecialCharTok{==}\ConstantTok{TRUE}\NormalTok{,}\StringTok{\textquotesingle{}Obs.Num\textquotesingle{}}\NormalTok{]]}
\NormalTok{vec\_y1\_ros}
\end{Highlighting}
\end{Shaded}

\begin{verbatim}
## numeric(0)
\end{verbatim}

\hypertarget{y2}{%
\subsection{Y2}\label{y2}}

Tidak outlier yang terlihat, Kemudian Disini maka cenderung menjulur ke
kanan

\begin{Shaded}
\begin{Highlighting}[]
\NormalTok{ros\_y2 }\OtherTok{\textless{}{-}} \FunctionTok{rosnerTest}\NormalTok{(y1,}\AttributeTok{k=}\DecValTok{5}\NormalTok{)}
\NormalTok{allstat\_ros }\OtherTok{\textless{}{-}}\NormalTok{ ros\_y1}\SpecialCharTok{$}\NormalTok{all.stats}
\NormalTok{vec\_y2\_ros }\OtherTok{\textless{}{-}}\NormalTok{ y1[allstat\_ros[allstat\_ros[}\StringTok{\textquotesingle{}Outlier\textquotesingle{}}\NormalTok{]}\SpecialCharTok{==}\ConstantTok{TRUE}\NormalTok{,}\StringTok{\textquotesingle{}Obs.Num\textquotesingle{}}\NormalTok{]]}
\NormalTok{vec\_y2\_ros}
\end{Highlighting}
\end{Shaded}

\begin{verbatim}
## numeric(0)
\end{verbatim}

\hypertarget{nomor-2}{%
\section{Nomor 2}\label{nomor-2}}

\hypertarget{x1}{%
\subsection{X1}\label{x1}}

\hypertarget{trimmed-mean}{%
\subsubsection{Trimmed Mean}\label{trimmed-mean}}

\begin{Shaded}
\begin{Highlighting}[]
\FunctionTok{mean}\NormalTok{(x1,}\AttributeTok{trim =} \FloatTok{0.05}\NormalTok{)}
\end{Highlighting}
\end{Shaded}

\begin{verbatim}
## [1] 21.02891
\end{verbatim}

\hypertarget{winsorzed-mean}{%
\subsubsection{Winsorzed Mean}\label{winsorzed-mean}}

\begin{Shaded}
\begin{Highlighting}[]
\FunctionTok{winsor.mean}\NormalTok{(x1, }\AttributeTok{trim=}\FloatTok{0.05}\NormalTok{)}
\end{Highlighting}
\end{Shaded}

\begin{verbatim}
## [1] 20.92647
\end{verbatim}

\hypertarget{m-estimators}{%
\subsubsection{M-Estimators}\label{m-estimators}}

\hypertarget{huber}{%
\paragraph{huber}\label{huber}}

\begin{Shaded}
\begin{Highlighting}[]
\NormalTok{fun\_hub\_x1 }\OtherTok{\textless{}{-}} \FunctionTok{psi.huber}\NormalTok{(x1, }\AttributeTok{k =} \FloatTok{1.345}\NormalTok{, }\AttributeTok{deriv =} \DecValTok{0}\NormalTok{)}
\NormalTok{df\_hub\_x1 }\OtherTok{\textless{}{-}} \FunctionTok{data.frame}\NormalTok{(}\AttributeTok{xx=}\NormalTok{x1,}\AttributeTok{psi\_hub=}\NormalTok{fun\_hub\_x1)}
\FunctionTok{ggplot}\NormalTok{(df\_hub\_x1,}\FunctionTok{aes}\NormalTok{(xx,psi\_hub)) }\SpecialCharTok{+} 
  \FunctionTok{geom\_line}\NormalTok{() }\SpecialCharTok{+} \FunctionTok{ggtitle}\NormalTok{(}\StringTok{"Function huber in X1"}\NormalTok{)}
\end{Highlighting}
\end{Shaded}

\includegraphics{G64190069_latihan-praktikum-12_files/figure-latex/unnamed-chunk-10-1.pdf}

\begin{Shaded}
\begin{Highlighting}[]
\NormalTok{M\_es\_hub }\OtherTok{\textless{}{-}} \FunctionTok{huber}\NormalTok{(x1, }\AttributeTok{k =} \FloatTok{1.345}\NormalTok{)}
\NormalTok{M\_es\_hub}\SpecialCharTok{$}\NormalTok{mu}
\end{Highlighting}
\end{Shaded}

\begin{verbatim}
## [1] 20.77871
\end{verbatim}

\hypertarget{x2-1}{%
\subsection{X2}\label{x2-1}}

\hypertarget{trimmed-mean-1}{%
\subsubsection{Trimmed Mean}\label{trimmed-mean-1}}

\begin{Shaded}
\begin{Highlighting}[]
\FunctionTok{mean}\NormalTok{(x2,}\AttributeTok{trim =} \FloatTok{0.05}\NormalTok{)}
\end{Highlighting}
\end{Shaded}

\begin{verbatim}
## [1] 2.020437
\end{verbatim}

\hypertarget{winsorzed-mean-1}{%
\subsubsection{Winsorzed Mean}\label{winsorzed-mean-1}}

\begin{Shaded}
\begin{Highlighting}[]
\FunctionTok{winsor.mean}\NormalTok{(x2, }\AttributeTok{trim=}\FloatTok{0.05}\NormalTok{)}
\end{Highlighting}
\end{Shaded}

\begin{verbatim}
## [1] 2.045956
\end{verbatim}

\hypertarget{m-estimators-1}{%
\subsubsection{M-Estimators}\label{m-estimators-1}}

\hypertarget{huber-1}{%
\paragraph{huber}\label{huber-1}}

\begin{Shaded}
\begin{Highlighting}[]
\NormalTok{fun\_hub\_x2 }\OtherTok{\textless{}{-}} \FunctionTok{psi.huber}\NormalTok{(x2, }\AttributeTok{k =} \FloatTok{1.345}\NormalTok{, }\AttributeTok{deriv =} \DecValTok{0}\NormalTok{)}
\NormalTok{df\_hub\_x2 }\OtherTok{\textless{}{-}} \FunctionTok{data.frame}\NormalTok{(}\AttributeTok{xx=}\NormalTok{x2,}\AttributeTok{psi\_hub=}\NormalTok{fun\_hub\_x2)}
\FunctionTok{ggplot}\NormalTok{(df\_hub\_x2,}\FunctionTok{aes}\NormalTok{(xx,psi\_hub)) }\SpecialCharTok{+} 
  \FunctionTok{geom\_line}\NormalTok{() }\SpecialCharTok{+} \FunctionTok{ggtitle}\NormalTok{(}\StringTok{"Function huber in X2"}\NormalTok{)}
\end{Highlighting}
\end{Shaded}

\includegraphics{G64190069_latihan-praktikum-12_files/figure-latex/unnamed-chunk-13-1.pdf}

\begin{Shaded}
\begin{Highlighting}[]
\NormalTok{M\_es\_hub }\OtherTok{\textless{}{-}} \FunctionTok{huber}\NormalTok{(x2, }\AttributeTok{k =} \FloatTok{1.345}\NormalTok{)}
\NormalTok{M\_es\_hub}\SpecialCharTok{$}\NormalTok{mu}
\end{Highlighting}
\end{Shaded}

\begin{verbatim}
## [1] 1.687373
\end{verbatim}

\hypertarget{y1-1}{%
\subsection{Y1}\label{y1-1}}

\hypertarget{trimmed-mean-2}{%
\subsubsection{Trimmed Mean}\label{trimmed-mean-2}}

\begin{Shaded}
\begin{Highlighting}[]
\FunctionTok{mean}\NormalTok{(y1,}\AttributeTok{trim =} \FloatTok{0.05}\NormalTok{)}
\end{Highlighting}
\end{Shaded}

\begin{verbatim}
## [1] 0.5074825
\end{verbatim}

\hypertarget{winsorzed-mean-2}{%
\subsubsection{Winsorzed Mean}\label{winsorzed-mean-2}}

\begin{Shaded}
\begin{Highlighting}[]
\FunctionTok{winsor.mean}\NormalTok{(y1, }\AttributeTok{trim=}\FloatTok{0.05}\NormalTok{)}
\end{Highlighting}
\end{Shaded}

\begin{verbatim}
## [1] 0.504838
\end{verbatim}

\hypertarget{m-estimators-2}{%
\subsubsection{M-Estimators}\label{m-estimators-2}}

\hypertarget{huber-2}{%
\paragraph{huber}\label{huber-2}}

\begin{Shaded}
\begin{Highlighting}[]
\NormalTok{fun\_hub\_y1 }\OtherTok{\textless{}{-}} \FunctionTok{psi.huber}\NormalTok{(y1, }\AttributeTok{k =} \FloatTok{1.345}\NormalTok{, }\AttributeTok{deriv =} \DecValTok{0}\NormalTok{)}
\NormalTok{df\_hub\_y1 }\OtherTok{\textless{}{-}} \FunctionTok{data.frame}\NormalTok{(}\AttributeTok{xx=}\NormalTok{y1,}\AttributeTok{psi\_hub=}\NormalTok{fun\_hub\_y1)}
\FunctionTok{ggplot}\NormalTok{(df\_hub\_y1,}\FunctionTok{aes}\NormalTok{(xx,psi\_hub)) }\SpecialCharTok{+} 
  \FunctionTok{geom\_line}\NormalTok{() }\SpecialCharTok{+} \FunctionTok{ggtitle}\NormalTok{(}\StringTok{"Function huber in Y1"}\NormalTok{)}
\end{Highlighting}
\end{Shaded}

\includegraphics{G64190069_latihan-praktikum-12_files/figure-latex/unnamed-chunk-16-1.pdf}

\begin{Shaded}
\begin{Highlighting}[]
\NormalTok{M\_es\_hub }\OtherTok{\textless{}{-}} \FunctionTok{huber}\NormalTok{(y1, }\AttributeTok{k =} \FloatTok{1.345}\NormalTok{)}
\NormalTok{M\_es\_hub}\SpecialCharTok{$}\NormalTok{mu}
\end{Highlighting}
\end{Shaded}

\begin{verbatim}
## [1] 0.5045575
\end{verbatim}

\hypertarget{y2-1}{%
\subsection{Y2}\label{y2-1}}

\hypertarget{trimmed-mean-3}{%
\subsubsection{Trimmed Mean}\label{trimmed-mean-3}}

\begin{Shaded}
\begin{Highlighting}[]
\FunctionTok{mean}\NormalTok{(y2,}\AttributeTok{trim =} \FloatTok{0.05}\NormalTok{)}
\end{Highlighting}
\end{Shaded}

\begin{verbatim}
## [1] 9.939848
\end{verbatim}

\hypertarget{winsorzed-mean-3}{%
\subsubsection{Winsorzed Mean}\label{winsorzed-mean-3}}

\begin{Shaded}
\begin{Highlighting}[]
\FunctionTok{winsor.mean}\NormalTok{(y2, }\AttributeTok{trim=}\FloatTok{0.05}\NormalTok{)}
\end{Highlighting}
\end{Shaded}

\begin{verbatim}
## [1] 10.11176
\end{verbatim}

\hypertarget{m-estimators-3}{%
\subsubsection{M-Estimators}\label{m-estimators-3}}

\hypertarget{huber-3}{%
\paragraph{huber}\label{huber-3}}

\begin{Shaded}
\begin{Highlighting}[]
\NormalTok{fun\_hub\_y2 }\OtherTok{\textless{}{-}} \FunctionTok{psi.huber}\NormalTok{(y2, }\AttributeTok{k =} \FloatTok{1.345}\NormalTok{, }\AttributeTok{deriv =} \DecValTok{0}\NormalTok{)}
\NormalTok{df\_hub\_y2 }\OtherTok{\textless{}{-}} \FunctionTok{data.frame}\NormalTok{(}\AttributeTok{xx=}\NormalTok{y2,}\AttributeTok{psi\_hub=}\NormalTok{fun\_hub\_y2)}
\FunctionTok{ggplot}\NormalTok{(df\_hub\_y2,}\FunctionTok{aes}\NormalTok{(xx,psi\_hub)) }\SpecialCharTok{+} 
  \FunctionTok{geom\_line}\NormalTok{() }\SpecialCharTok{+} \FunctionTok{ggtitle}\NormalTok{(}\StringTok{"Function huber in Y2"}\NormalTok{)}
\end{Highlighting}
\end{Shaded}

\includegraphics{G64190069_latihan-praktikum-12_files/figure-latex/unnamed-chunk-19-1.pdf}

\begin{Shaded}
\begin{Highlighting}[]
\NormalTok{M\_es\_hub }\OtherTok{\textless{}{-}} \FunctionTok{huber}\NormalTok{(y2, }\AttributeTok{k =} \FloatTok{1.345}\NormalTok{)}
\NormalTok{M\_es\_hub}\SpecialCharTok{$}\NormalTok{mu}
\end{Highlighting}
\end{Shaded}

\begin{verbatim}
## [1] 9.890736
\end{verbatim}

\hypertarget{nomor-3}{%
\section{Nomor 3}\label{nomor-3}}

IQR

Berdasarkan
\href{https://web.ipac.caltech.edu/staff/fmasci/home/astro_refs/RobustEstimators.pdf}{dari
link ini}, Maka Sebelum kita menentukan nilai IQR dari suatu data, maka
kita lihat distribusi apa yang cocok dengan data tersebut

\hypertarget{x1-1}{%
\subsection{X1}\label{x1-1}}

\begin{Shaded}
\begin{Highlighting}[]
\FunctionTok{descdist}\NormalTok{(x1)}
\end{Highlighting}
\end{Shaded}

\includegraphics{G64190069_latihan-praktikum-12_files/figure-latex/unnamed-chunk-20-1.pdf}

\begin{verbatim}
## summary statistics
## ------
## min:  19.4353   max:  24.75342 
## median:  20.51857 
## mean:  21.02891 
## estimated sd:  1.537735 
## estimated skewness:  1.728735 
## estimated kurtosis:  6.605395
\end{verbatim}

Dengan Menggunakan Graph Cullen and Frey graph, Maka kemungkinan
distribusi x1 antara beta dan gamma. Tetapi dikarenakan error
menggunakan Beta, maka bisa kita cek gamma. Tetapi sebelum itu, kita
gunakan qqplot dahulu yaaa dengan distibusi normal terlebih dahulu

\begin{Shaded}
\begin{Highlighting}[]
\CommentTok{\# qqnorm(x1, pch = 1, frame = FALSE)}
\CommentTok{\# qqline(x1, col = "steelblue", lwd = 2)}
\FunctionTok{qqPlot}\NormalTok{(x1)}
\end{Highlighting}
\end{Shaded}

\includegraphics{G64190069_latihan-praktikum-12_files/figure-latex/unnamed-chunk-21-1.pdf}

\begin{verbatim}
## [1] 10  2
\end{verbatim}

Setelah diperhatikan, Sebenarnya distribusi ini mendekati normal, tetapi
dikarenakan adanya outlier, maka jika kita test dengan ks.test

\begin{Shaded}
\begin{Highlighting}[]
\NormalTok{fit\_x1 }\OtherTok{\textless{}{-}} \FunctionTok{fitdist}\NormalTok{(x1,}\StringTok{"norm"}\NormalTok{)}
\FunctionTok{summary}\NormalTok{(fit\_x1)}
\end{Highlighting}
\end{Shaded}

\begin{verbatim}
## Fitting of the distribution ' norm ' by maximum likelihood 
## Parameters : 
##       estimate Std. Error
## mean 21.028911  0.4613204
## sd    1.458823  0.3262021
## Loglikelihood:  -17.96569   AIC:  39.93137   BIC:  40.53654 
## Correlation matrix:
##      mean sd
## mean    1  0
## sd      0  1
\end{verbatim}

\begin{Shaded}
\begin{Highlighting}[]
\FunctionTok{plot}\NormalTok{(fit\_x1)}
\end{Highlighting}
\end{Shaded}

\includegraphics{G64190069_latihan-praktikum-12_files/figure-latex/unnamed-chunk-22-1.pdf}

\begin{Shaded}
\begin{Highlighting}[]
\FunctionTok{ks.test}\NormalTok{(x1,}\StringTok{"pnorm"}\NormalTok{,}\AttributeTok{mean=}\FloatTok{21.028911}\NormalTok{,}\AttributeTok{sd=}\FloatTok{1.458823}\NormalTok{)}
\end{Highlighting}
\end{Shaded}

\begin{verbatim}
## 
##  One-sample Kolmogorov-Smirnov test
## 
## data:  x1
## D = 0.20699, p-value = 0.7123
## alternative hypothesis: two-sided
\end{verbatim}

\begin{Shaded}
\begin{Highlighting}[]
\FunctionTok{shapiro.test}\NormalTok{(x1)}
\end{Highlighting}
\end{Shaded}

\begin{verbatim}
## 
##  Shapiro-Wilk normality test
## 
## data:  x1
## W = 0.83965, p-value = 0.0437
\end{verbatim}

jika diperhatikan, jika ks.test normal, sedangkan shapiro-wilk test ini
tidak normal, untuk p-value shapiro-wilk ini, mendekati normal, meskipun
mendekati p-value.

\textbf{X1 itu adalah Distribusi Normal (mean=21.028911,sd=1.458823)}

\hypertarget{x2-2}{%
\subsection{X2}\label{x2-2}}

\begin{Shaded}
\begin{Highlighting}[]
\FunctionTok{descdist}\NormalTok{(x2)}
\end{Highlighting}
\end{Shaded}

\includegraphics{G64190069_latihan-praktikum-12_files/figure-latex/unnamed-chunk-24-1.pdf}

\begin{verbatim}
## summary statistics
## ------
## min:  -2.991378   max:  7.526452 
## median:  1.468013 
## mean:  2.020437 
## estimated sd:  2.802018 
## estimated skewness:  0.3648262 
## estimated kurtosis:  4.365285
\end{verbatim}

Disini, banyak sekali kemungkinan yaaa. jika diperhatikan, distribusinya
ada logistic , atau mungkin lognormal. Kita cek qqplot yang logistic
yaaa

\begin{Shaded}
\begin{Highlighting}[]
\FunctionTok{qqPlot}\NormalTok{(x2,}\StringTok{"logis"}\NormalTok{)}
\end{Highlighting}
\end{Shaded}

\includegraphics{G64190069_latihan-praktikum-12_files/figure-latex/unnamed-chunk-25-1.pdf}

\begin{verbatim}
## [1]  9 10
\end{verbatim}

Kemudian, kita lihat, bahwa x2 distribusi lognormal , jika diperhatikan,
bahwa ada outlier yaaa. ada 2 outlier yaa, yang lain bisa dikatakan
distribusi logistik

\begin{Shaded}
\begin{Highlighting}[]
\NormalTok{fit\_x2 }\OtherTok{\textless{}{-}} \FunctionTok{fitdist}\NormalTok{(x2,}\StringTok{"logis"}\NormalTok{)}
\NormalTok{fit\_x2}
\end{Highlighting}
\end{Shaded}

\begin{verbatim}
## Fitting of the distribution ' logis ' by maximum likelihood 
## Parameters:
##          estimate Std. Error
## location 1.882585  0.7817310
## scale    1.446234  0.3932554
\end{verbatim}

\begin{Shaded}
\begin{Highlighting}[]
\FunctionTok{plot}\NormalTok{(fit\_x2)}
\end{Highlighting}
\end{Shaded}

\includegraphics{G64190069_latihan-praktikum-12_files/figure-latex/unnamed-chunk-26-1.pdf}
Bisa dilihat bahwa sebenarnya data ini merupakan distribusi logistik

\begin{Shaded}
\begin{Highlighting}[]
\FunctionTok{ks.test}\NormalTok{(x2,}\StringTok{"plogis"}\NormalTok{,}\AttributeTok{location=}\FloatTok{1.882585}\NormalTok{,}\AttributeTok{scale=}\FloatTok{1.446234}\NormalTok{)}
\end{Highlighting}
\end{Shaded}

\begin{verbatim}
## 
##  One-sample Kolmogorov-Smirnov test
## 
## data:  x2
## D = 0.19993, p-value = 0.7491
## alternative hypothesis: two-sided
\end{verbatim}

\begin{Shaded}
\begin{Highlighting}[]
\FunctionTok{shapiro.test}\NormalTok{(x2)}
\end{Highlighting}
\end{Shaded}

\begin{verbatim}
## 
##  Shapiro-Wilk normality test
## 
## data:  x2
## W = 0.93129, p-value = 0.4607
\end{verbatim}

jika diperhatikan, jika ks.test logis, sedangkan shapiro-wilk test ini
tidak normal, berarti tepat bahwa data ini distribusi logistik dan juga
bukan distribusi normal

\textbf{X2 itu adalah Distribusi Logistik
(location=1.882585,scale=1.446234)}

\hypertarget{y1-2}{%
\subsection{Y1}\label{y1-2}}

\begin{Shaded}
\begin{Highlighting}[]
\FunctionTok{descdist}\NormalTok{(y1)}
\end{Highlighting}
\end{Shaded}

\includegraphics{G64190069_latihan-praktikum-12_files/figure-latex/unnamed-chunk-28-1.pdf}

\begin{verbatim}
## summary statistics
## ------
## min:  -0.499999   max:  0.9888917 
## median:  0.5144103 
## mean:  0.4906979 
## estimated sd:  0.3482005 
## estimated skewness:  -0.5900393 
## estimated kurtosis:  3.094567
\end{verbatim}

Bisa diperhatikan, bahwa Asumsi distribusi uniform, normal , gamma, dan
logistik. berarti kita cek yahhh 4 distribusi, berarti kita cek yaaa
dahulu uniform

\begin{Shaded}
\begin{Highlighting}[]
\FunctionTok{qqPlot}\NormalTok{(y1,}\AttributeTok{distribution=}\StringTok{"unif"}\NormalTok{)}
\end{Highlighting}
\end{Shaded}

\includegraphics{G64190069_latihan-praktikum-12_files/figure-latex/unnamed-chunk-29-1.pdf}

\begin{verbatim}
## [1] 100  99
\end{verbatim}

Jika diperhatikan, ada beberapa partikel yang melewati batas partikel
yaaa. nah itu outlier yaaa, ada sedikit kesalahan disitu, kemudian
hasilnya pada qqplot sudah pasti adalah distribusi uniform dikarenakan
sudah tepat

\begin{Shaded}
\begin{Highlighting}[]
\NormalTok{fit\_y1 }\OtherTok{\textless{}{-}} \FunctionTok{fitdist}\NormalTok{(y1,}\StringTok{"unif"}\NormalTok{)}
\NormalTok{fit\_y1}
\end{Highlighting}
\end{Shaded}

\begin{verbatim}
## Fitting of the distribution ' unif ' by maximum likelihood 
## Parameters:
##       estimate Std. Error
## min -0.4999990         NA
## max  0.9888917         NA
\end{verbatim}

\begin{Shaded}
\begin{Highlighting}[]
\FunctionTok{plot}\NormalTok{(fit\_y1)}
\end{Highlighting}
\end{Shaded}

\includegraphics{G64190069_latihan-praktikum-12_files/figure-latex/unnamed-chunk-30-1.pdf}

\begin{Shaded}
\begin{Highlighting}[]
\FunctionTok{ks.test}\NormalTok{(y1,}\StringTok{"punif"}\NormalTok{)}
\end{Highlighting}
\end{Shaded}

\begin{verbatim}
## 
##  One-sample Kolmogorov-Smirnov test
## 
## data:  y1
## D = 0.072115, p-value = 0.6758
## alternative hypothesis: two-sided
\end{verbatim}

\begin{Shaded}
\begin{Highlighting}[]
\FunctionTok{shapiro.test}\NormalTok{(y1)}
\end{Highlighting}
\end{Shaded}

\begin{verbatim}
## 
##  Shapiro-Wilk normality test
## 
## data:  y1
## W = 0.94574, p-value = 0.0004397
\end{verbatim}

\textbf{Y1 itu Distribusi Uniform (min=-0,4999990,max=0.3464552)}

\hypertarget{y2-2}{%
\subsection{Y2}\label{y2-2}}

\begin{Shaded}
\begin{Highlighting}[]
\FunctionTok{descdist}\NormalTok{(y2)}
\end{Highlighting}
\end{Shaded}

\includegraphics{G64190069_latihan-praktikum-12_files/figure-latex/unnamed-chunk-32-1.pdf}

\begin{verbatim}
## summary statistics
## ------
## min:  2.279965   max:  48.86305 
## median:  9.499855 
## mean:  10.82127 
## estimated sd:  7.081105 
## estimated skewness:  3.10021 
## estimated kurtosis:  16.07243
\end{verbatim}

bisa diperhatikan bahwa observasi ini, bisa gamma atau beta
distribusinya, tapi kita cek dulu normal deh

\begin{Shaded}
\begin{Highlighting}[]
\FunctionTok{fitdist}\NormalTok{(y2,}\StringTok{"gamma"}\NormalTok{)}
\end{Highlighting}
\end{Shaded}

\begin{verbatim}
## Fitting of the distribution ' gamma ' by maximum likelihood 
## Parameters:
##        estimate Std. Error
## shape 3.6125932 0.48912413
## rate  0.3338428 0.04849241
\end{verbatim}

\begin{Shaded}
\begin{Highlighting}[]
\FunctionTok{plot}\NormalTok{(}\FunctionTok{fitdist}\NormalTok{(y2,}\StringTok{"gamma"}\NormalTok{))}
\end{Highlighting}
\end{Shaded}

\includegraphics{G64190069_latihan-praktikum-12_files/figure-latex/unnamed-chunk-33-1.pdf}

\begin{Shaded}
\begin{Highlighting}[]
\DocumentationTok{\#\# hubungan anatara gamma denganchisquare adalah sebagai berikut https://programmathically.com/chi{-}square{-}distribution{-}and{-}degrees{-}of{-}freedom/}
\FunctionTok{ks.test}\NormalTok{(y2,}\StringTok{"pgamma"}\NormalTok{,}\AttributeTok{shape=}\DecValTok{5}\NormalTok{,}\AttributeTok{rate=}\FloatTok{0.5}\NormalTok{)}
\end{Highlighting}
\end{Shaded}

\begin{verbatim}
## 
##  One-sample Kolmogorov-Smirnov test
## 
## data:  y2
## D = 0.04966, p-value = 0.9661
## alternative hypothesis: two-sided
\end{verbatim}

dikarenakan shape = 5 dan juga rate = 0.5 maka bisa dikatakan bahwa
shape =5 dengan syarat rate=0.5 --\textgreater{} df = 10

\begin{Shaded}
\begin{Highlighting}[]
\FunctionTok{ks.test}\NormalTok{(y2,}\StringTok{"pchisq"}\NormalTok{,}\AttributeTok{df=}\DecValTok{10}\NormalTok{)}
\end{Highlighting}
\end{Shaded}

\begin{verbatim}
## 
##  One-sample Kolmogorov-Smirnov test
## 
## data:  y2
## D = 0.04966, p-value = 0.9661
## alternative hypothesis: two-sided
\end{verbatim}

Setelah diujicoba dan diperhatikan dengan seksama , maka hasilnya adalah

\textbf{Y2 in distribution Chi Square (df=10)}

\hypertarget{perhitungan-iqr}{%
\section{Perhitungan IQR}\label{perhitungan-iqr}}

Rujukan adalah
\href{https://dc.etsu.edu/cgi/viewcontent.cgi?article=2187\&context=etd}{link
ini ya}

\begin{enumerate}
\def\labelenumi{\arabic{enumi}.}
\tightlist
\item
  X1 Distribusi Normal (mean=21.028911,sd=1.458823)
\item
  X2 Distribusi Logistik (location=1.882585,scale=1.446234)
\item
  Y1 Distribusi Uniform (min=-0,4999990,max=0.3464552)
\item
  Y2 Distribusi Chisquare (df=10)
\end{enumerate}

\hypertarget{x1-2}{%
\subsection{X1}\label{x1-2}}

Nilai IQR dari X1 adalah

\begin{Shaded}
\begin{Highlighting}[]
\DocumentationTok{\#\# Dikarenakan Normal maka Formulanya }
\DocumentationTok{\#\# IQRnorm ≈ 1.34900σ}
\NormalTok{IQR\_X1 }\OtherTok{\textless{}{-}} \FloatTok{1.349}\SpecialCharTok{*}\FunctionTok{sd}\NormalTok{(x1)}
\FunctionTok{print}\NormalTok{(}\FunctionTok{paste}\NormalTok{(}\StringTok{\textquotesingle{}IQR\_X1 =\textquotesingle{}}\NormalTok{,IQR\_X1,}\AttributeTok{sep=}\StringTok{\textquotesingle{} \textquotesingle{}}\NormalTok{))}
\end{Highlighting}
\end{Shaded}

\begin{verbatim}
## [1] "IQR_X1 = 2.0744039770859"
\end{verbatim}

\hypertarget{x2-3}{%
\subsection{X2}\label{x2-3}}

Nilai IQR dari X2 adalah

\begin{Shaded}
\begin{Highlighting}[]
\DocumentationTok{\#\# Dikarenakan Logistic maka Formulanya }
\NormalTok{IQR\_X2 }\OtherTok{\textless{}{-}} \FunctionTok{IQR}\NormalTok{(x2)}
\FunctionTok{print}\NormalTok{(}\FunctionTok{paste}\NormalTok{(}\StringTok{\textquotesingle{}IQR\_X2 =\textquotesingle{}}\NormalTok{,IQR\_X2,}\AttributeTok{sep=}\StringTok{\textquotesingle{} \textquotesingle{}}\NormalTok{))}
\end{Highlighting}
\end{Shaded}

\begin{verbatim}
## [1] "IQR_X2 = 2.52318627238694"
\end{verbatim}

\hypertarget{y1-3}{%
\subsection{Y1}\label{y1-3}}

Nilai IQR dari Y1 adalah

\begin{Shaded}
\begin{Highlighting}[]
\DocumentationTok{\#\# Dikarenakan Logistic maka Formulanya }
\DocumentationTok{\#\# IQRunif ≈ 0.5(θ2 {-} θ1)}
\NormalTok{IQR\_Y1 }\OtherTok{\textless{}{-}} \FloatTok{0.5}\SpecialCharTok{*}\NormalTok{(}\FloatTok{0.3464552}\SpecialCharTok{{-}}\NormalTok{(}\SpecialCharTok{{-}}\FloatTok{0.4999990}\NormalTok{))}
\FunctionTok{print}\NormalTok{(}\FunctionTok{paste}\NormalTok{(}\StringTok{\textquotesingle{}IQR\_Y1 =\textquotesingle{}}\NormalTok{,IQR\_Y1,}\AttributeTok{sep=}\StringTok{\textquotesingle{} \textquotesingle{}}\NormalTok{))}
\end{Highlighting}
\end{Shaded}

\begin{verbatim}
## [1] "IQR_Y1 = 0.4232271"
\end{verbatim}

\hypertarget{y2-3}{%
\subsection{Y2}\label{y2-3}}

Nilai IQR dari Y2 adalah

\begin{Shaded}
\begin{Highlighting}[]
\DocumentationTok{\#\# Dikarenakan Logistic maka Formulanya }
\DocumentationTok{\#\# IQRunif ≈ 2(e\^{}µ)sinh(0.6745σ)}
\NormalTok{IQR\_Y2 }\OtherTok{\textless{}{-}} \FunctionTok{IQR}\NormalTok{(y2)}
\FunctionTok{print}\NormalTok{(}\FunctionTok{paste}\NormalTok{(}\StringTok{\textquotesingle{}IQR\_Y2 =\textquotesingle{}}\NormalTok{,IQR\_Y2,}\AttributeTok{sep=}\StringTok{\textquotesingle{} \textquotesingle{}}\NormalTok{))}
\end{Highlighting}
\end{Shaded}

\begin{verbatim}
## [1] "IQR_Y2 = 6.51586843556726"
\end{verbatim}

Jika diperhatikan bahwa hasil IQR itu sendiri adalah tidak berpengaruh
terhadap pencilan dikarenakan data tersebut adalah pada quartil 1 dan 3,
sehingga pencilan tidak dianggap.

\end{document}
